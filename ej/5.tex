\section{Proyecciones $\Lambda_+$ y $\Lambda_-$}
\begingroup\color{blue}
Prueba que
\begin{equation}
\Lambda_+=\frac{\cp+m}{2m},\qquad
\Lambda_-=\frac{\cp-m}{2m}
\end{equation}
satisfacen
\begin{equation}
\Lambda_\pm^2=\Lambda_\pm,\qquad
\Lambda_+\Lambda_-=0.
\end{equation}
¿Cómo actúan estos proyectores sobre los espinores básicos
\begin{equation}
u_r(k)=u(r,k),\qquad
v_r(k)=v(r,k)\;\text{?}
\end{equation}
\endgroup
Desarrollando,
\begin{equation}\begin{split}
\Lambda_+\Lambda_+
&=\frac1{4m^2}(p_\mu\gamma^\mu+m)(p_\nu\gamma^\nu+m)\\
&=\frac1{4m^2}(p_\mu p_\nu\gamma^\mu\gamma^\nu+2mp_\mu\gamma^\mu+m^2)\\
&=\frac1{4m^2}(\tfrac12p_\mu p_\nu\{\gamma^\mu,\gamma^\nu\}+2mp_\mu\gamma^\mu+m^2)\\
&=\frac1{4m^2}(p_\mu p^\mu+2mp_\mu\gamma^\mu+m^2)\\
&=\frac1{4m^2}(2mp_\mu\gamma^\mu+2m^2)\\
&=\frac1{2m}(p_\mu\gamma^\mu+m)\\
&=\Lambda_+
\end{split}\end{equation}
También,
\begin{equation}\begin{split}
\Lambda_-\Lambda_-
&=\frac1{4m^2}(p_\mu\gamma^\mu-m)(p_\nu\gamma^\nu-m)\\
&=\frac1{4m^2}(p_\mu p_\nu\gamma^\mu\gamma^\nu-2mp_\mu\gamma^\mu+m^2)\\
&=\frac1{4m^2}(\tfrac12p_\mu p_\nu\{\gamma^\mu,\gamma^\nu\}-2mp_\mu\gamma^\mu+m^2)\\
&=\frac1{4m^2}(p_\mu p^\mu-2mp_\mu\gamma^\mu+m^2)\\
&=\frac1{4m^2}(-2mp_\mu\gamma^\mu+2m^2)\\
&=-\frac1{2m}(p_\mu\gamma^\mu-m)\\
&=-\Lambda_-
\end{split}\end{equation}
quizás me equivoqué. Si no, el signo se resuelve redefiniendo al operador $\Lambda_-$.
$$\Lambda_-\mapsto-\Lambda_-=-\frac{p-m}{2m}$$
\par Para conocer los efectos de estos proyectores con los espinores $u$ y $v$ usaremos la información de las notas TCC.2.6, p. 357, y del problema 4.b de esta tarea para reescribir las $\Lambda$ como:
\begin{equation}
\Lambda_+=\frac1{2m}u^s(p)\bar u_s(p),
\qquad
\Lambda_-=-\frac1{2m}v^s(p)\bar v_s(p).
\end{equation}
Por lo tanto,
\begin{equation}
\Lambda_+u^r(p)
=\frac1{2m}u^s(p)\bar u_s(p)u^r(p)
=\frac{2m}{2m}\delta^r_su^s(p)
=u^r(p),
\end{equation}
donde hemos usado que el producto matricial es asociativo.
\begin{equation}
\Lambda_+v^r(p)
=\frac1{2m}u^s(p)\bar u_s(p)v^r(p) = 0,
\quad
\text{porque $\bar u_s(p)v^r(p)=0$.}
\end{equation}
En este desarrollo utilizamos el segundo resultado del problema 4.c.
\begin{equation}
\Lambda_-u^r(p)
=-\frac1{2m}v^s(p)\bar v_s(p)u^r(p)
=0,
\quad
\text{porque $\bar v_s(p)u^r(p)=0$.}
\end{equation}
En este otro, utilizamos el primer resultado del problema 4.c.
\begin{equation}
\Lambda_-v^r(p)=-\frac1{2m}v^s(p)\bar v_s(p)v^r(p)
=-\frac{-2m}{2m}\delta_s^rv^s(p)
=v^r(p)
\end{equation}












