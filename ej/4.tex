\section{Productos $\bar vv$, $v\bar v$ y $\bar vu$}
\begingroup\color{blue}
Muestre que
\begin{equation}
\begin{split}
\text{a)}\quad&\bar v^r(k)v^s(k)=-2m\delta^{rs}\\
\text{b)}\quad&\sum_{s=1}^2v(p,s)\bar v(p,s)=\cp-m\\
\text{c)}\quad&\bar v(p,r)u(p,s)=0
\end{split}
\end{equation}
\endgroup
\subsection{}
De la ecuación (\ref{pera}) tenemos:
\begin{equation*}\begin{split}
\bar v^r(k)v^s(k)&=v^{r\dagger}(k)\gamma^0v^s(k)\\
&=
\left(
-\frac{v^{r\dagger}(k_R)(m-\ck^\dagger)}{\sqrt{2m(k_0+m)}}
\right)\gamma^0
\left(
-\frac{(m-\ck)v^s(k_R)}{\sqrt{2m(k_0+m)}}
\right)\\
&=\frac1{2(k_0+m)}(\eta^{r\dagger}\quad-\eta^{r\dagger})
(m-\ck^\dagger)\gamma^0(m-\ck)
\left(\begin{array}r
\eta^s\\-\eta^s
\end{array}\right)\\
&=\frac1{2(k_0+m)}(\eta^{r\dagger}\quad-\eta^{r\dagger})\gamma^0
(m-\ck)(m-\ck)
\left(\begin{array}r
\eta^s\\-\eta^s
\end{array}\right)\\
&=\frac1{2(k_0+m)}(-\eta^{r\dagger}\quad\eta^{r\dagger})
(m^2-2m\ck+k_\mu k^\mu)
\left(\begin{array}r
\eta^s\\-\eta^s
\end{array}\right)
\end{split}\end{equation*}
\begin{equation}\begin{split}
&=\frac{2m}{2(k_0+m)}(-\eta^{r\dagger}\quad\eta^{r\dagger})
(m-\ck)
\left(\begin{array}r
\eta^s\\-\eta^s
\end{array}\right)\\
&=\frac{2m}{2(k_0+m)}(-\eta^{r\dagger}\quad\eta^{r\dagger})
(m-\ck)
\left(\begin{array}r
\eta^s\\-\eta^s
\end{array}\right)\\
&=\frac{2m}{2(k_0+m)}(\quad
-2m\eta^{r\dagger}\eta^s
-\eta^{r\dagger}k_\mu\sigma_+^\mu\eta^s
-\eta^{r\dagger}k_\mu\sigma_-^\mu\eta^s\quad)\\
&=\frac{-2m}{2(k_0+m)}(2m+2k_0)\eta^{r\dagger}\eta^s\\
&=-2m\delta^{rs}
\end{split}\end{equation}
\subsection{}
Usaremos la convención de Einstein para el ínidce del espín.
\begin{equation}
\begin{split}
&v^s(k)\bar v_s(k)\\
&\quad=v^s(k)v_s^\dagger(k)\gamma^0\\
&\quad=
\left(
-\frac{(m-\ck)v^s(k_R)}{\sqrt{2m(k_0+m)}}
\right)
\left(
-\frac{v_s^\dagger(k_R)(m-\ck^\dagger)}{\sqrt{2m(k_0+m)}}
\right)\gamma^0\\
&\quad=
\frac1{2(k_0+m)}
(m-\ck)
\left(
\begin{array}r
\eta^s\!\!\\
\!\!-\eta^s\!\!
\end{array}
\right)
(\eta_s^\dagger\quad-\!\eta_s^\dagger)
(m-\ck^\dagger)\gamma^0\\
&\quad=
\frac1{2(k_0+m)}
(m-\ck)
\left(\begin{array}{rr}
\eta^s\eta_s^\dagger&-\eta^s\eta_s^\dagger\!\!\\
\!\!-\eta^s\eta_s^\dagger&\eta^s\eta_s^\dagger\!\!
\end{array}\right)\gamma^0
(m-\ck)\gamma^0\gamma^0\\
&\quad=
\frac1{2(k_0+m)}
(m-\ck)
\left(\begin{array}{rr}
\id&-\id\!\\
\!\!\!-\id&\id\!
\end{array}\right)
\gamma^0
(m-\ck)\\
&\quad=
\frac1{2(k_0+m)}
(m-\ck)
(\id-\gamma^0)
\gamma^0(m-\ck)\\
&\quad=
\frac1{2(k_0+m)}
(m-\ck)
(\gamma^0-\id)
(m-\ck)
\end{split}
\label{ragatoba}
\end{equation}
Desarrollaremos dos expresiones de manera independiente.
\begin{equation}
(m-\ck)(-\id)(m-\ck)
=-(m^2-2m\ck+k_\mu k^\mu)
=2m(\ck-m)
\label{garabato}
\end{equation}
Por otro lado,
\begin{equation}\begin{split}
&(m-\ck)\gamma^0(m-\ck)\\
&\quad=
\begin{pmatrix}
m&-k_\mu\sigma_-^\mu\\
-k_\mu\sigma_+^\mu&m
\end{pmatrix}
\begin{pmatrix}0&\id\\\id&0\end{pmatrix}
\begin{pmatrix}
m&-k_\mu\sigma_-^\mu\\
-k_\mu\sigma_+^\mu&m
\end{pmatrix}\\
&\quad=
\begin{pmatrix}
m&-k_\mu\sigma_-^\mu\\
-k_\mu\sigma_+^\mu&m
\end{pmatrix}
\begin{pmatrix}
-k_\mu\sigma_+^\mu&m\\
m&-k_\mu\sigma_-^\mu
\end{pmatrix}\\
&\quad=
\begin{pmatrix}
-mk_\mu\sigma_+^\mu -mk_\mu\sigma_-^\mu &
m^2+k_\mu k_\nu\sigma_-^\mu\sigma_-^\nu \\
m^2+k_\mu k_\nu\sigma_+^\mu\sigma_+^\nu &
-mk_\mu\sigma_+^\mu -mk_\mu\sigma_-^\mu
\end{pmatrix}
\end{split}\label{public}\end{equation}
Una vez más, dividiremos el trabajo en dos casos.
\begin{equation}
-mk_\mu\sigma_+^\mu -mk_\mu\sigma_-^\mu
=-mk_\mu(\sigma_+^\mu+\sigma_-^\mu)
=-2mk_0
\end{equation}
Por otro lado tenemos:
\begin{equation}\begin{split}
m^2+k_\mu k_\nu\sigma_\pm^\mu\sigma_\pm^\nu
&=m^2+\frac12k_\mu k_\nu\{\sigma_\pm^\mu,\sigma_\pm^\nu\}\\
&=m^2+2k_0k_\mu\sigma_\pm^\mu-k_\mu k^\mu\\
&=2k_0k_\mu\sigma_\pm^\mu,
\end{split}\end{equation}
donde hemos usado:
\begin{equation}\begin{split}
\frac12\{\sigma_\pm^\mu,\sigma_\pm^\nu\}
&=\begin{pmatrix}
\id&\sigma_\pm^x&\sigma_\pm^y&\sigma_\pm^z\\
\sigma_\pm^x&\id&0&0\\
\sigma_\pm^y&0&\id&0\\
\sigma_\pm^z&0&0&\id
\end{pmatrix}\\
&=\delta_0^\nu\sigma_\pm^\mu+\delta_0^\mu\sigma_\pm^\nu-\eta^{\mu\nu}\id.
\end{split}\label{gordo}\end{equation}
Incorporando estos resultados a la matriz de (\ref{public}) obtenemos:
\begin{equation}\begin{split}
(m-\ck)\gamma^0(m-\ck)
&=
\begin{pmatrix}
-2mk_0&2k_0k_\mu\sigma_-^\mu\\
2k_0k_\mu\sigma_+^\mu&-2mk_0
\end{pmatrix}=2k_0(\ck-m)
\end{split}\end{equation}
Este resultado y el (\ref{garabato}) los incorporamos a (\ref{ragatoba}) y obtenemos:
\begin{equation}\begin{split}
v^s(k)\bar v_s(k)
&=\frac1{2(k_0+m)}(m-\ck)(\gamma^0-\id)(m-\ck)\\
&=\frac1{2(k_0+m)}(2m (\ck-m) + 2k_0(\ck-m))\\
&=\ck-m,
\end{split}\end{equation}
que es lo que queríamos demostrar.
\subsection{}
Expresando la $\bar v$ y la $u$ en términos de los espinores en reposo:
\begin{equation}\begin{split}
\bar v^r(k)u^s(k)
&=v^{r\dagger}(k)\gamma^0u^s(k)\\
&=\left(
\frac{
v^{r\dagger}(k_R)(\ck^\dagger-m)
}{
\sqrt{2m(k_0+m)}
}
\right)
\gamma^0
\left(
\frac{
(\ck+m)u^s(k_R)
}{
\sqrt{2m(k_0+m)}
}
\right)\\
&=\frac{v^{r\dagger}(k_R)\gamma^0(\ck-m)(\ck+m)u^s(k_R)}{2m(k_0+m)}\\
&=\frac{v^{r\dagger}(k_R)\gamma^0(k_\mu k^\mu -m^2)u^s(k_R)}{2m(k_0+m)}\\
&=0,\quad\text{porque $m^2=k_\mu k^\mu$.}
\end{split}\end{equation}
Un resultado que utilizaremos en el problema 5 tiene un desarrollo similar:
\begin{equation}\begin{split}
\bar u^r(k)v^s(k)
&=u^{r\dagger}(k)\gamma^0v^s(k)\\
&=\left(
\frac{
u^{r\dagger}(k_R)(\ck^\dagger+m)
}{
\sqrt{2m(k_0+m)}
}
\right)
\gamma^0
\left(
\frac{
(\ck+m)v^s(k_R)
}{
\sqrt{2m(k_0-m)}
}
\right)\\
&=\frac{u^{r\dagger}(k_R)\gamma^0(\ck+m)(\ck-m)v^s(k_R)}{2m(k_0+m)}\\
&=\frac{u^{r\dagger}(k_R)\gamma^0(k_\mu k^\mu -m^2)v^s(k_R)}{2m(k_0+m)}\\
&=0,\quad\text{por la misma razón.}
\end{split}\end{equation}
